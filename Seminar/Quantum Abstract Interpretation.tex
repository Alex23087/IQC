\documentclass[11pt,svgnames,smaller,aspectratio=169,english]{beamer}

\usepackage[utf8]{inputenc}
\usepackage{tikz}
\usetikzlibrary{graphs,graphdrawing,arrows.meta,positioning,quotes,external,chains,graphs.standard}
\tikzexternalize[prefix=tikzfigures/]
\usegdlibrary{trees}
\usepackage{xcolor}
\usepackage{subfig}
\usepackage{braket}

\usetheme{UniPi}
\setbeamercovered{transparent=0}
\setbeamersize{text margin left=15mm,text margin right=15mm}
\setbeamertemplate{itemize subitem}{-}
\uselanguage{english}
\languagepath{english}

\newcommand{\specialcell}[2][c]{%
  \begin{tabular}[#1]{@{}c@{}}#2\end{tabular}}

\def\mathunderline#1#2{\color{#1}\underline{{\color{black}#2}}\color{black}}
\newcommand{\tensor}{\otimes}




\title{Quantum Abstract Interpretation}
\author{Alessandro Scala}
\institute[Università di Pisa]{Università di Pisa\\Dipartimento di Informatica}
\date{Pisa, 24 Luglio 2023}
\extra{Seminar for the \textbf{Introduction to Quantum Computing} course}

\usepackage{transparent}
\mode<presentation>

\begin{document}

\begin{frame} 
	\titlepage
\end{frame}
\logo{\transparent{0.2}\includegraphics[height=2cm]{images/cherubino}}



% ------------------ CHANGE FROM HERE ------------------

\begin{frame}{Roadmap}
	\tableofcontents
\end{frame}

\begin{frame}{Introduction}
	As quantum computing advances, we would like to have some means to prove correctness properties on quantum programs, \emph{especially} since quantum programming is counterintuitive.\\
\end{frame}

\begin{frame}<1-4>[label=reasons]{Reasons}
	The naive way to check properties of a program is to run it and \textbf<2>{observe} its behaviour.\\\;\\
	\pause
	\alert{We cannot \textbf{observe} the state of a quantum program!}\\\;\\
	\pause
	Could \textbf{simulation} on a classical machine solve this issue?\\\;\\
	\pause
	\alert{No: \textbf{exponential} space and time cost.}\\\;\\\;\\
	\pause
	\begin{center}
		\large \color{blue} Solution: abstract interpretation
	\end{center}
\end{frame}

\begin{frame}[noframenumbering]{Example}
	\only<1>{
		\begin{gather*}
			n_{qubits} = 1\\\;\\
			\ket{0}\bra{0}\\\;\\
			\begin{pmatrix}
				1 & 0\\0 & 0
			\end{pmatrix}\\\;\\
			2^2 = 4 \text{ complex numbers}
		\end{gather*}
	}
	\only<2>{
		\begin{gather*}
			n_{qubits} = 2\\\;\\
			\ket{00}\bra{00}\\\;\\
			\begin{pmatrix}
				1 & 0 & 0 & 0 \\0 & 0 & 0 & 0 \\0 & 0 & 0 & 0 \\0 & 0 & 0 & 0	
			\end{pmatrix}\\\;\\
			2^4 = 16 \text{ complex numbers}
		\end{gather*}
	}

	\only<3>{
		\setcounter{MaxMatrixCols}{8}
		\begin{gather*}
			n_{qubits} = 4\\\;\\
			\ket{0000}\bra{0000}\\\;\\
			\begin{pmatrix}
				1 & 0 & 0 & 0 & 0 & 0 & 0 & 0\\
				0& 0 & 0 & 0 & 0 & 0 & 0 & 0 \\
				0& 0 & 0 & 0 & 0 & 0 & 0 & 0 \\
				0& 0 & 0 & 0 & 0 & 0 & 0 & 0 \\
				0& 0 & 0 & 0 & 0 & 0 & 0 & 0 \\
				0& 0 & 0 & 0 & 0 & 0 & 0 & 0 \\
				0& 0 & 0 & 0 & 0 & 0 & 0 & 0 \\
				0& 0 & 0 & 0 & 0 & 0 & 0 & 0
			\end{pmatrix}\\\;\\
			2^8 = 64 \text{ complex numbers}
		\end{gather*}
	}

	\only<4->{
		\begin{gather*}
			n_{qubits} = 300\\\;\\
			\ket{0}^{\tensor_{300}}\bra{0}^{\tensor_{300}}\\\;\\
			?????\\\;\\
			\only<5->{2^{600} = 414951556888099295851240786369116115101244623224243689999\\
			56573296906528114129081463997070489471037942881978866113007\\
			89182395151075411775307886874834113963687061181803401509523685376\\\;\\
			\text{\alert{Bigger than the number of atoms in the universe.}}}
		\end{gather*}
	}
\end{frame}

\againframe<4->[noframenumbering]{reasons}

\begin{frame}{Ingredients}
	\begin{itemize}
		\item Abstract domain
		\begin{itemize}
			\item Abstraction function
			\item Concretization function
			\item Abstract operations
		\end{itemize}
		\item Assertions
	\end{itemize}
\end{frame}

\begin{frame}{Density Matrix}
	Instead of dealing with a state $\ket{\phi}$ in vector form, we use their \emph{density matrix}:
	\begin{equation*}
		\ket{\phi}\bra{\phi} \text{ (For a pure state)}		
	\end{equation*}
	\uncover<2->{
		{\large \color{blue} Example:}
		\begin{align*}
			\ket{\beta_{00}} &= \frac{\ket{00} + \ket{11}}{\sqrt{2}}\\
			\ket{\beta_{00}}\bra{\beta_{00}} &= \frac{1}{2}\begin{pmatrix}
				1 & 0 & 0 & 1\\
				0 & 0 & 0 & 0\\
				0 & 0 & 0 & 0\\
				1 & 0 & 0 & 1
			\end{pmatrix}
		\end{align*}
	}
\end{frame}

\begin{frame}{Reduced Density Matrix}
	Suppose we have a composite quantum system $AB = A \tensor B$, and we want to consider a state $\ket{\phi} \in AB$ with respect to the subsystem $A$.
	\begin{gather*}
		A = \mathbb{C}^{2^n}\times\mathbb{C}^{2^n}\\
		B = \mathbb{C}^{2^m}\times\mathbb{C}^{2^m}\\
		AB = (\mathbb{C}^{2^n}\times\mathbb{C}^{2^n})\tensor(\mathbb{C}^{2^m}\times\mathbb{C}^{2^m})
	\end{gather*}
	\begin{align*}
		Tr_B[\rho] &: AB \rightarrow A && Tr_A[\rho] : AB \rightarrow B\\
		Tr_B[\rho] &= \sum_{v=0}^{2^m} (I_A \tensor \bra{v})\rho(I_A \tensor \ket{v}) && Tr_A[\rho] = \sum_{v=0}^{2^n} (\bra{v} \tensor I_B)\rho(\ket{v} \tensor I_B)
	\end{align*}
\end{frame}
% Introduction
% Reasons
% Ingredients
% Density matrices
% Abstract state space
% Galois connections
% Abstract operations
% Assertions
% Conclusions

\end{document}




% n_{qubits} = 5\\
% \ket{00000}\bra{00000}\\
% % \begin{pmatrix}
% % 	1 & 0 & 0 & 0 & 0 & 0 & 0 & 0 & 0 & 0 & 0 & 0 & 0 & 0 & 0 & 0 \\
% % 	0 & 0 & 0 & 0 & 0 & 0 & 0 & 0 & 0 & 0 & 0 & 0 & 0 & 0 & 0 & 0 \\
% % 	0 & 0 & 0 & 0 & 0 & 0 & 0 & 0 & 0 & 0 & 0 & 0 & 0 & 0 & 0 & 0 \\
% % 	0 & 0 & 0 & 0 & 0 & 0 & 0 & 0 & 0 & 0 & 0 & 0 & 0 & 0 & 0 & 0 \\
% % 	0 & 0 & 0 & 0 & 0 & 0 & 0 & 0 & 0 & 0 & 0 & 0 & 0 & 0 & 0 & 0 \\
% % 	0 & 0 & 0 & 0 & 0 & 0 & 0 & 0 & 0 & 0 & 0 & 0 & 0 & 0 & 0 & 0 \\
% % 	0 & 0 & 0 & 0 & 0 & 0 & 0 & 0 & 0 & 0 & 0 & 0 & 0 & 0 & 0 & 0 \\
% % 	0 & 0 & 0 & 0 & 0 & 0 & 0 & 0 & 0 & 0 & 0 & 0 & 0 & 0 & 0 & 0 \\
% % 	0 & 0 & 0 & 0 & 0 & 0 & 0 & 0 & 0 & 0 & 0 & 0 & 0 & 0 & 0 & 0 \\
% % 	0 & 0 & 0 & 0 & 0 & 0 & 0 & 0 & 0 & 0 & 0 & 0 & 0 & 0 & 0 & 0 \\
% % 	0 & 0 & 0 & 0 & 0 & 0 & 0 & 0 & 0 & 0 & 0 & 0 & 0 & 0 & 0 & 0 \\
% % 	0 & 0 & 0 & 0 & 0 & 0 & 0 & 0 & 0 & 0 & 0 & 0 & 0 & 0 & 0 & 0 \\
% % 	0 & 0 & 0 & 0 & 0 & 0 & 0 & 0 & 0 & 0 & 0 & 0 & 0 & 0 & 0 & 0 \\
% % 	0 & 0 & 0 & 0 & 0 & 0 & 0 & 0 & 0 & 0 & 0 & 0 & 0 & 0 & 0 & 0 \\
% % 	0 & 0 & 0 & 0 & 0 & 0 & 0 & 0 & 0 & 0 & 0 & 0 & 0 & 0 & 0 & 0 \\
% % 	0 & 0 & 0 & 0 & 0 & 0 & 0 & 0 & 0 & 0 & 0 & 0 & 0 & 0 & 0 & 0
% % \end{pmatrix}