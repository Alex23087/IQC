\documentclass[11pt,svgnames,smaller,aspectratio=43,english]{beamer}

\usepackage[utf8]{inputenc}
\usepackage{tikz}
\usetikzlibrary{graphs,graphdrawing,arrows.meta,positioning,quotes,external,chains,graphs.standard}
%\tikzexternalize[prefix=tikzfigures/]
\usegdlibrary{trees}
\usepackage{xcolor}
\usepackage{subfig}
\usepackage{braket}
\usepackage{spot}

\usetheme{UniPi}
\setbeamercovered{transparent=0}
\setbeamersize{text margin left=15mm,text margin right=15mm}
\setbeamertemplate{itemize subitem}{-}
\uselanguage{english}
\languagepath{english}

\newcommand{\specialcell}[2][c]{%
  \begin{tabular}[#1]{@{}c@{}}#2\end{tabular}}

\def\mathunderline#1#2{\color{#1}\underline{{\color{black}#2}}\color{black}}
\newcommand{\tensor}{\otimes}

\newcommand{\bunder}[2]{\underbrace{#1}_{\text{\scriptsize\tabular[t]{@{}c@{}}#2\endtabular}}}

\renewcommand{\implies}{\Rightarrow}
\renewcommand{\iff}{\Leftrightarrow}



\title{Quantum Abstract Interpretation}
\author{Alessandro Scala}
\institute[Università di Pisa]{Università di Pisa\\Dipartimento di Informatica}
\date{Pisa, 24 Luglio 2023}
\extra{Seminar for the \textbf{Introduction to Quantum Computing} course}

\usepackage{transparent}
\mode<presentation>

\begin{document}

\begin{frame} 
	\titlepage
\end{frame}
\logo{\transparent{0.2}\includegraphics[height=2cm]{images/cherubino}}



% ------------------ CHANGE FROM HERE ------------------

\AtBeginSection[]
{
	\begin{frame}{Roadmap}
		\tableofcontents[currentsection]
	\end{frame}
}

\section{Introduction}
\begin{frame}{Introduction}
	As quantum computing advances, we would like to have some means to prove correctness properties on quantum programs, \emph{especially} since quantum programming is unintuitive.\\
\end{frame}

\subsection{Reasons}
\begin{frame}<1-4>[label=reasons]{Reasons}
	The naive way to check properties of a program is to run it and \textbf<2->{observe} its behaviour.\\\;\\
	\pause
	\alert{We \textbf{cannot observe} the state of a quantum program!}\\\;\\
	\pause
	Could \textbf{simulation} on a classical machine solve this issue?\\\;\\
	\pause
	\alert{No: \textbf{exponential} space and time cost.}\\\;\\
	\pause
	Static checking of programs by hand?\\\;\\
	\pause
	{\color{orange} Yes, but time consuming. Needs to be adapted to the specific program.}\\\;\\
	\pause
	\begin{center}
		\large \color{blue} Solution: abstract interpretation
	\end{center}
\end{frame}

\begin{frame}[noframenumbering]{Example}
	\only<1>{
		\begin{gather*}
			n_{qubits} = 1\\\;\\
			\ket{0}\bra{0}\\\;\\
			\begin{pmatrix}
				1 & 0\\0 & 0
			\end{pmatrix}\\\;\\
			2^2 = 4 \text{ complex numbers}
		\end{gather*}
	}
	\only<2>{
		\begin{gather*}
			n_{qubits} = 2\\\;\\
			\ket{00}\bra{00}\\\;\\
			\begin{pmatrix}
				1 & 0 & 0 & 0 \\0 & 0 & 0 & 0 \\0 & 0 & 0 & 0 \\0 & 0 & 0 & 0	
			\end{pmatrix}\\\;\\
			2^4 = 16 \text{ complex numbers}
		\end{gather*}
	}

	\only<3>{
		\setcounter{MaxMatrixCols}{8}
		\begin{gather*}
			n_{qubits} = 3\\\;\\
			\ket{000}\bra{000}\\\;\\
			\begin{pmatrix}
				1 & 0 & 0 & 0 & 0 & 0 & 0 & 0\\
				0& 0 & 0 & 0 & 0 & 0 & 0 & 0 \\
				0& 0 & 0 & 0 & 0 & 0 & 0 & 0 \\
				0& 0 & 0 & 0 & 0 & 0 & 0 & 0 \\
				0& 0 & 0 & 0 & 0 & 0 & 0 & 0 \\
				0& 0 & 0 & 0 & 0 & 0 & 0 & 0 \\
				0& 0 & 0 & 0 & 0 & 0 & 0 & 0 \\
				0& 0 & 0 & 0 & 0 & 0 & 0 & 0
			\end{pmatrix}\\\;\\
			2^6 = 64 \text{ complex numbers}
		\end{gather*}
	}

	\only<4->{
		\begin{gather*}
			n_{qubits} = 300\\\;\\
			\ket{0}^{\tensor_{300}}\bra{0}^{\tensor_{300}}\\\;\\
			\only<4>{?????\\\;\\}
			\only<5->{2^{600} = 41495155688809929585124078636911611510124462322424368\\9999
			56573296906528114129081463997070489471037942881978866113\\007
			891823951510754117753078868748341139636870611818034015095\\23685376\\\;\\
			\text{\alert{Bigger than the number of atoms in the universe.}}}
		\end{gather*}
	}
\end{frame}

\againframe<4->[noframenumbering]{reasons}

\subsection{Abstract Interpretation}
\begin{frame}{What is abstract interpretation?}
	Instead of considering the \textbf{concrete domain}, we restrict our analysis to a \textbf{more coarse domain}, an \textbf{abstract domain}.\\
	\vspace*{0.5em}
	\pause
	The abstract domain we consider should be:
	\vspace*{-0.5em}
	\pause
	\begin{center}
		\begin{columns}
			\begin{column}[T]{0.95\textwidth}
				\begin{itemize}[<+->]
					\setlength{\itemindent}{4em}
					\item[fine enough] to capture the information we are interested in (e.g. the state of the quantum system belongs to the span of two vectors)
					\item[as coarse as possible] to discard information we are not interested in, and thus be substantially more efficient to be analyzed
				\end{itemize}
			\end{column}
		\end{columns}
	\end{center}
	\uncover<+->{
		Abstract interpretation is usually sound, but not complete:
		\begin{itemize}
		\item AI returns \textbf{true} $\implies$ Property is \textbf{true}
		\item AI returns \textbf{false} $\implies$ Property can be either \textbf{true} or \textbf{false}		
		\end{itemize}
	}
\end{frame}

\begin{frame}{Ingredients}
	\begin{itemize}[<+->]
		\item Abstract domain
		\begin{itemize}[<+->]
			\item Abstraction function: from more concrete to more abstract domain
			\item Concretization function: from more abstract to more concrete domain
			\item Abstract operations: to represent concrete operations in the abstract domain
		\end{itemize}
		\item Assertions: properties we can prove with abstract interpretation
	\end{itemize}
\end{frame}

\section{Preliminaries}
\subsection{Density Matrix}
\begin{frame}{Density Matrix}
	Instead of dealing with a state $\ket{\phi}$ in vector form, we use its \emph{density matrix}:

	\begin{columns}
		\begin{column}[T]{0.35\textwidth}
			\begin{equation*}
				\rho_\phi = \ket{\phi}\bra{\phi} \text{ (For a pure state)}		
			\end{equation*}
		\end{column}
		\begin{column}[T]{0.5\textwidth}
			\begin{itemize}
				\item positive semi-definite
				\item $Tr(\rho) = 1$
				\item projection ($P = P^\dagger = P^2$)
			\end{itemize}
		\end{column}
	\end{columns}

	\vspace*{1em}

	\uncover<+->{}
	\uncover<+->{
		{\large \color{blue} Example:}
		\begin{align*}
			\ket{\beta_{00}} &= \frac{\ket{00} + \ket{11}}{\sqrt{2}}\\
			\rho_{\beta_{00}} &= \ket{\beta_{00}}\bra{\beta_{00}}\uncover<+->{
				= \frac{1}{2} (\ket{00} + \ket{11})(\bra{00} + \bra{11})\\
			}
			\uncover<+->{
				&= \frac{1}{2} (\ket{00}\bra{00} + \ket{00}\bra{11} + \ket{11}\bra{00} + \ket{11}\bra{11})\\
			}
			\uncover<+->{
				&= \frac{1}{2} \begin{pmatrix}
					1 & 0 & 0 & 1\\
					0 & 0 & 0 & 0\\
					0 & 0 & 0 & 0\\
					1 & 0 & 0 & 1
				\end{pmatrix}
			}
		\end{align*}
	}
	
\end{frame}

\subsection{Reduced Density Matrix}
\begin{frame}{Reduced Density Matrix}
	\uncover<+->{
		Suppose we have a composite quantum system $AB = A \tensor B$, and we want to focus our attention on a state $\ket{\phi} \in AB$ with respect to subsystem $A$.
	}
	\uncover<+->{
		\begin{gather*}
			A = \mathbb{C}^{2^n}\times\mathbb{C}^{2^n}\quad
			B = \mathbb{C}^{2^m}\times\mathbb{C}^{2^m}\\
			AB = (\mathbb{C}^{2^n}\times\mathbb{C}^{2^n})\tensor(\mathbb{C}^{2^m}\times\mathbb{C}^{2^m})
		\end{gather*}
	}
	\vspace*{-2em}
	\uncover<+->{
		\begin{align*}
			&Tr_B[\rho] : AB \rightarrow A && Tr_A[\rho] : AB \rightarrow B\\
			&Tr_B[\alpha\tensor\beta] = \alpha \cdot Tr(\beta) &&
			Tr_A[\alpha\tensor\beta] = Tr(\alpha) \cdot \beta
		\end{align*}
		\vspace*{-2em}
		\begin{gather*}
			\left.\begin{array}{@{}ll@{}}
				Tr_S[\rho + \sigma] = Tr_S[\rho] + Tr_S[\sigma]\\
				Tr_S[a \cdot \rho] = a \cdot Tr_S[\rho]
			\end{array}\right\} \text{ (Linearity)}
		\end{gather*}
	}

	\uncover<+->{
		Partial trace $Tr_B[\rho]$ \textbf{traces out} subsystem $B$.
	}
\end{frame}

% \begin{frame}{Example}
% 	\begin{align*}
% 		Tr_B[\rho] &= \sum_{v=0}^{2^m} (I_A \tensor \bra{v})\rho(I_A \tensor \ket{v}) &&&
% 		\rho_{\beta_{00}} = \frac{1}{2} \begin{pmatrix}
% 			1 & 0 & 0 & 1\\
% 			0 & 0 & 0 & 0\\
% 			0 & 0 & 0 & 0\\
% 			1 & 0 & 0 & 1
% 		\end{pmatrix}
% 	\end{align*}
% 	\begin{align*}
% 		Tr_B[\rho_{\beta_{00}}] =& (I_2\tensor\bra{0})\rho_{\beta_{00}}(I_2\tensor\ket{0}) + (I_2\tensor\bra{1})\rho_{\beta_{00}}(I_2\tensor\ket{1})\\
% 		\only<1>{
% 			=& \frac{1}{2} \begin{pmatrix}
% 				1 & 0 & 0 & 0 \\ 0 & 0 & 1 & 0
% 			\end{pmatrix}\begin{pmatrix}
% 				1 & 0 & 0 & 1\\
% 				0 & 0 & 0 & 0\\
% 				0 & 0 & 0 & 0\\
% 				1 & 0 & 0 & 1
% 			\end{pmatrix}\begin{pmatrix}
% 				1 & 0 \\ 0 & 0 \\ 0 & 1 \\ 0 & 0
% 			\end{pmatrix}+\\
% 			&\frac{1}{2} \begin{pmatrix}
% 				0 & 1 & 0 & 0 \\ 0 & 0 & 0 & 1
% 			\end{pmatrix}\begin{pmatrix}
% 				1 & 0 & 0 & 1\\
% 				0 & 0 & 0 & 0\\
% 				0 & 0 & 0 & 0\\
% 				1 & 0 & 0 & 1
% 			\end{pmatrix}\begin{pmatrix}
% 				0 & 0 \\ 1 & 0 \\ 0 & 0 \\ 0 & 1
% 			\end{pmatrix}
% 		}
% 		\uncover<2->{
% 			=& \frac{1}{2} \begin{pmatrix}
% 				1 & 0 \\ 0 & 0
% 			\end{pmatrix} + \frac{1}{2} \begin{pmatrix}
% 				0 & 0 \\ 0 & 1
% 			\end{pmatrix}\\
% 		}
% 		\uncover<3->{
% 			=& \frac{1}{2} \begin{pmatrix}
% 				1 & 0 \\ 0 & 1
% 			\end{pmatrix}\\
% 		}
% 		\uncover<4->{
% 			=& \frac{\ket{0}\bra{0} + \ket{1}\bra{1}}{2}\;\\\;\\\;\\\;\\
% 		}
% 	\end{align*}
% \end{frame}

\begin{frame}{Example}
	\begin{gather*}
		A = \mathbb{C}^2\times \mathbb{C}^2\quad B = \mathbb{C}^2\times \mathbb{C}^2\quad AB = A\tensor B\\
		\rho_{\beta_{00}} = \ket{\beta_{00}}\bra{\beta_{00}} = \frac{\ket{00}\bra{00} + \ket{00}\bra{11} + \ket{11}\bra{00} + \ket{11}\bra{11}}{2}
	\end{gather*}
	\vspace*{2em}
	\begin{align*}
		\hspace*{-1cm}
		Tr_B[\rho_{\beta_{00}}] =& \uncover<2->{\frac{\left(Tr_B[\ket{00}\bra{00}] + Tr_b[\ket{00}\bra{11}] + Tr_b[\ket{11}\bra{00}] + Tr_b[\ket{11}\bra{11}]\right)}{2}}\\
		\uncover<3->{=& \frac{(\ket{0}\bra{0} \cdot \spot*<4>[rectangle,fill=blue!30]{\braket{0|0}}) + (\ket{0}\bra{1} \cdot \spot*<4>[rectangle,fill=red!30]{\braket{0|1}}) + (\ket{1}\bra{0} \cdot \spot*<4>[rectangle,fill=red!30]{\braket{1|0}}) + (\ket{1}\bra{1} \cdot \spot*<4>[rectangle,fill=blue!30]{\braket{1|1}})}{2}}\\
		\uncover<5->{=& \frac{\ket{0}\bra{0} + \ket{1}\bra{1}}{2}}
	\end{align*}
\end{frame}

\begin{frame}{Loss of precision}
	\alert{Computing a reduced density matrix \textbf{discards information}!}
	\begin{align*}
		\rho_{\beta_{00}} =& \frac{\ket{00}\bra{00} + \ket{00}\bra{11} + \ket{11}\bra{00} + \ket{11}\bra{11}}{2} & \text{(Pure state)}\\
		\rho_2 =& \frac{\ket{00}\bra{00} + \ket{01}\bra{01} + \ket{10}\bra{10} + \ket{11}\bra{11}}{4} & \text{(Mixed state)}\\\;\\
		\uncover<2->{
			&Tr_B[\rho_{\beta_{00}}] = \frac{\ket{0}\bra{0} + \ket{1}\bra{1}}{2} = Tr_B[\rho_2]
		}
	\end{align*}
	\;\\
	\uncover<2->{
		% A reduced density matrix can be obtained starting from two different states.\\\;\\
		The partial traces of two different initial states can be equal.\\\;\\
		For a state $\rho\in A\tensor B$, even if we know $Tr_B[\rho]$ and $Tr_A[\rho]$, we cannot uniquely determine $\rho$. 
	}
\end{frame}

\begin{frame}{Projection Subspaces} %TODO: Work on this
	Each projection $P$ corresponds to a linear subspace $S_P = \{v \mid Pv = v\}$.\\
	\;\\
	\pause
	The support of a matrix $P$	is the subspace orthogonal to its kernel, i.e., the set $supp(P) = \{v \mid Pv \neq 0\}$.\\
	\;\\
	\pause
	This duality between projections and subspaces will be employed multiple times and will often be implied.
\end{frame}

\section{Abstract Domain}
\begin{frame}{Abstract Domain}
	\begin{gather*}
		\uncover<+->{
			\mathcal{D} = \mathbb{C}^{2^n}\tensor\mathbb{C}^{2^n},\quad S = (s_1, ..., s_m),\quad 1 \leq m \leq 2^n,\quad s_i \subseteq [n]\\
		}
		\uncover<+->{
			\mathit{AbsDom}(S) = \left\{(P_{s_1}, ..., P_{s_m}) \mid P_{s_i} \text{ is a projection in } \mathbb{C}^{2^{|s_i|}}\tensor\mathbb{C}^{2^{|s_i|}}\right\}
		}
	\end{gather*}
	\uncover<+->{
		Intuitively, given a tuple $S$ of sets of qubits, an abstract state $\overline{\sigma} \in \mathit{AbsDom}(S)$ is a tuple of projections over those qubits.\\\;\\
	}
	\uncover<+->{
		{\color{blue} Special case:}
		\begin{gather*}
			T = ([n]) \implies \mathit{AbsDom}(T) \simeq \mathcal{D}
		\end{gather*}
	}
\end{frame}

\begin{frame}{Fineness Relation} %TODO: Explain why finer
	Let $S = (s_1, ..., s_m)$ and $T = (t_1, ..., t_m)$ (with $1 \leq m \leq 2^n$), then:
	\begin{equation*}
		\bunder{S \trianglelefteq T}{``T is finer than S''} \triangleq \forall i \in [m].\; s_i \subseteq t_i
	\end{equation*}
	$T$ is ``more concrete'' than S.\\\;\\
	\pause
	% Minimal elements: tuples $(a_1, ... a_m)$ where all $a_i$ are singleton sets.\\
	Least element: $\bot = (\emptyset, ..., \emptyset)$.\\
	Greatest element: $\top = ([n], ... [n]) = [n]^m$.\\\;\\

	\pause
	$\mathit{AbsDom}(\bot)$ corresponds to a state so abstract that it holds no information at all.\\ %TODO: Maybe remove?
	$\mathit{AbsDom}(\top)$ corresponds to tuples where every projection is a concrete state.
\end{frame}

\subsection{Abstraction and Concretization Functions}
\begin{frame}{Abstraction Function}
	\uncover<+->{
		\vspace*{-2em}
		\begin{gather*}
			S \trianglelefteq T \triangleq \forall i \in [m].\; s_i \subseteq t_i\\
			\alpha_{T \rightarrow S}: \mathit{AbsDom}(T) \rightarrow \mathit{AbsDom}(S)
		\end{gather*}
	}
	\uncover<+->{
	\vspace*{-2em}
		\begin{gather*}
			\alpha_{T \rightarrow S}(Q_{t_1}, ..., Q_{t_{m}}) = (P_{s_1}, ..., P_{s_m})\\
			P_{s_i} = \bigcap_{t_j.\; s_i \subseteq t_J} supp(Tr_{t_j \setminus s_i} [Q_{t_j}])
		\end{gather*}
	}
	\uncover<+->{
		\\\vspace*{1em}
		Given an abstract state $\overline{\tau} \in \mathit{AbsDom}(T) = (Q_{t_1}, ..., Q_{t_m})$, we want to compute $\overline{\sigma} \in \mathit{AbsDom}(S) = (P_{s_1}, ..., P_{s_m})$. For each $i\in [m]$:
	}
	\begin{enumerate}[<+->]
		\item Find all $Q_{t_j}$s such that $s_i \subseteq t_j$. We know that at least one exists (for $j=i$), since $S \trianglelefteq T$.
		\item For each $Q_{t_j}$ found, trace out the qubits in $t_j$ that are not in $s_i$.
		\item Compute the support of the traced matrices (to preserve the structure of projections).
		\item Compute the intersection of the supports.
	\end{enumerate}
\end{frame}

\begin{frame}{Concretization Function}
	\uncover<+->{
		\vspace*{-2em}
		\begin{gather*}
			S \trianglelefteq T \triangleq \forall i \in [m].\; s_i \subseteq t_i\\
			\gamma_{S \rightarrow T}: \mathit{AbsDom}(S) \rightarrow \mathit{AbsDom}(T)
		\end{gather*}
	}
	\uncover<+->{
		\vspace*{-2em}
		\begin{gather*}
			\gamma_{S \rightarrow T}(P_{s_1}, ..., P_{s_m}) = (Q_{t_1}, ..., Q_{t_{m}})\\
			Q_{t_j} = \bigcap_{s_i.\; s_i \subseteq t_J} P_{s_i} \tensor I_{t_j \setminus s_i}
		\end{gather*}
	}
	\uncover<+->{
		\\\vspace*{1em}
		Given an abstract state $\overline{\sigma} \in \mathit{AbsDom}(S) = (P_{s_1}, ..., P_{s_m})$, we want to compute $\overline{\tau} \in \mathit{AbsDom}(T) = (Q_{t_1}, ..., Q_{t_m})$. For each $j\in [m]$:
	}
	\begin{enumerate}[<+->]
		\item Find all $P_{s_i}$s such that $s_i \subseteq t_j$. We know at least one exists (for $i=j$), since $S \trianglelefteq T$.
		\item Extend the projection to the space of all qubits in $t_j$, by computing the tensor product with the identity matrix.
		\item Compute the intersection of the extended projections.
	\end{enumerate}
\end{frame}

\subsection{Abstract Operations}
\begin{frame}{Abstract Operations}
	Operations on the concrete domain are unitary operators $U$.\\
	To apply them to a concrete state $\rho$ we compute $U\rho U^\dagger$.\\
	We want to define $U^\sharp: \mathit{AbsDom}(S) \rightarrow \mathit{AbsDom}(S)$.\\
	\pause
	\vspace*{1em}
	Let $T = (t_1, ..., t_m)$ s.t. $t_i = s_i \cup s_U$, where $s_U$ denotes the set of qubits that matrix $U$ acts on.\\
	\pause
	Let $U^{cg}(\overline{\sigma}) = (UT_{t_1}U^\dagger, ..., UT_{t_m}U^\dagger)$\pause , then
	\begin{equation*}
		U^\sharp = \alpha_{T \rightarrow S} \circ U^{cg} \circ \gamma_{S \rightarrow T}
	\end{equation*}
\end{frame}

\begin{frame}{Focus}
	\uncover<+->{}
	\begin{equation*}
		U^\sharp = \alpha_{T \rightarrow S} \circ U^{cg} \circ \gamma_{S \rightarrow T}
	\end{equation*}
	\uncover<+->{
		\resizebox{\linewidth}{!}{
			\begin{tikzpicture}[every node/.style={draw=none}]
				\node[] (A) {$\mathit{AbsDom}(S)$};
				\node[right=1cm of A] (B) {$\mathit{AbsDom}(T)$};
				\node[right=1cm of B] (C) {$\mathit{AbsDom}(T)$};
				\node[right=1cm of C] (D) {$\mathit{AbsDom}(S)$};

				\path[->] (A) edge node[above]{$\gamma_{S \rightarrow T}$} node[below]{\scriptsize Focus} (B);
				\path[->] (B) edge node[above]{$U^{cg}$} node[below]{\scriptsize Apply} (C);
				\path[->] (C) edge node[above]{$\alpha_{T \rightarrow S}$} node[below]{\scriptsize Unfocus} (D);
			\end{tikzpicture}
		}
	}
	\begin{itemize}[<+->]
		\item[Focus] - concretize to a new, finer domain with sufficient precision to accurately represent the operation ($\forall i \in [m].\; s_U \subseteq t_i$);
		\item[Apply] - apply the unitary operator to all the elements in the tuple
		\item[Unfocus] - abstract back to the original abstract domain to keep the representation more compact
	\end{itemize}
\end{frame}

\subsection{Assertions}
\begin{frame}{Assertions}
	We want to check that a state of a program lies in the span of two vectors.\\
	\pause
	We define an assertion as:
	\begin{equation*}
		A = span\{v_1 = \ket{a_1}...\ket{a_n},\;\; v_2 = \ket{b_1}...\ket{b_n}\}
	\end{equation*}
	\pause
	And a projection $proj(A)$ onto this subspace, such that:
	\begin{equation*}
		proj(A)v_1 = v_1 \qquad proj(A)v_2 = v_2
	\end{equation*}
\end{frame}


\begin{frame}{Order Relation on Abstract States}
	\begin{gather*}
		1 \leq m \leq 2^n,\quad S = (s_1, ... s_m),\quad \forall i \in [m].\; s_i \subseteq [n]\\
		\overline{\sigma} \in \mathit{AbsDom}(S) = (P_{s_1}, ..., P_{s_m}),\quad \overline{\tau} \in \mathit{AbsDom}(S) = (Q_{s_1}, ..., Q_{s_m})
	\end{gather*}
	\begin{gather*}
		\overline{\sigma} \sqsubseteq \overline{\tau} \triangleq \forall i \in [m].\; \hspace*{-1.9em}\bunder{P_{s_i} \subseteq Q_{s_i}}{Subspace interpretation\\of projections}
	\end{gather*}
\end{frame}

\begin{frame}{Monotonicity and Galois Connection}
	Monotonicity of \emph{abstraction}, \emph{concretization}, and \emph{abstract operations}:
	\vspace*{-1em}
	\begin{gather*}
		S \trianglelefteq T\\
		\forall \overline{\sigma}, \overline{\tau} \in \mathit{AbsDom}(T).\\
		\left(\begin{array}{@{}cc@{}}
			\overline{\sigma} \sqsubseteq \overline{\tau} \implies \alpha_{T\rightarrow S}(\overline{\sigma})\sqsubseteq\alpha_{T\rightarrow S}(\overline{\tau})\\
			\land\\
			\overline{\sigma} \sqsubseteq \overline{\tau} \implies \gamma_{T\rightarrow S}(\overline{\sigma})\sqsubseteq\gamma_{T\rightarrow S}(\overline{\tau})\\
			\land\\
			\overline{\sigma} \sqsubseteq \overline{\tau} \implies U^{\sharp}(\overline{\sigma})\sqsubseteq U^{\sharp}(\overline{\tau})
		\end{array}\right)
	\end{gather*}
	\pause
	Galois connection:
	\begin{gather*}
		\forall \overline{\sigma}\in \mathit{AbsDom}(S).\; \forall \overline{\tau}\in \mathit{AbsDom}([n]^m).\\
		\overline{\tau} \sqsubseteq \gamma_{S\rightarrow [n]^m}(\overline{\sigma}) \quad\iff\quad \alpha_{[n]^m \rightarrow S}(\overline{\tau}) \sqsubseteq \overline{\sigma}
	\end{gather*}
\end{frame}

\begin{frame}{Connectivity}
	\begin{gather*}
		S = (s_1, ..., s_m)\\
		S \text{ is connected} \triangleq \forall k \in [n-1].\; \exists r \in [m].\; k \in s_r \land k+1 \in s_r
	\end{gather*}

	\uncover<2->{
		{\color{blue} \large Example:}
		\begin{columns}
			\begin{column}[T]{0.7\linewidth}
				\begin{gather*}
					n = 5,\quad m = 3\\
					S = ({\color<5-7>{blue}\{0, 2, 3\}}, {\color<3-7>{green}\{0, 1, 2\}}, {\color<6-7>{red}\{3, 4, 5\}})
				\end{gather*}
			\end{column}
			\begin{column}[T]{0.3\linewidth}
				\begin{gather*}
					\only<3>{0,1 \in \{0,1,2\}}
					\only<4>{1,2 \in \{0,1,2\}}
					\only<5>{2,3 \in \{0,2,3\}}
					\only<6>{3,4 \in \{3,4,5\}}
					\only<7>{4,5 \in \{3,4,5\}}
				\end{gather*}
			\end{column}
		\end{columns}
	}
	\vspace*{1em}
	\uncover<8->{
		For an assertion $A$, if $S$ is connected, then:
		\begin{equation*}
			proj(A) = \gamma_{S\rightarrow [n]^m}(\alpha_{[n]^m \rightarrow S}(proj(A)))
		\end{equation*}
	}
\end{frame}

\begin{frame}{Assertion Checking}
	Given an assertion $A$, if the final state of a computation is $v$ and the final abstract state of the abstract interpretation is $\overline{v} \in \mathit{AbsDom}(S)$, with $S$ connected, then:\\
	\begin{gather*}
		\overline{v} \sqsubseteq \alpha_{[n]^m \rightarrow S}(proj(A)) \quad\implies\quad v \in A 
	\end{gather*}\\
	\vspace*{1.5em}
	\pause
	Essentially, if the abstract state satisfies the assertion, then the concrete one does as well.
\end{frame}

\begin{frame}{Choice of Tuple S}
	Fundamental choice for the shape of the abstract domain and, in turn, the success of the analysis.\\
	% \begin{columns}
	% 	\begin{column}[T]{0.3\textwidth}
	% 		\begin{gather*}
	% 			\text{Less, smaller sets}\\
	% 			\Downarrow\\
	% 			\text{\tabular{@{}cc@{}}
	% 				Less computational cost\\and memory footprint
	% 			\endtabular}
	% 		\end{gather*}
	% 	\end{column}
	% 	\begin{column}[T]{0.3\textwidth}
	% 		\begin{gather*}
	% 			\text{More, bigger sets}\\
	% 			\Downarrow\\
	% 			\text{\tabular{@{}cc@{}}
	% 				Less computational cost\\and memory footprint
	% 			\endtabular}
	% 		\end{gather*}
	% 	\end{column}
	% \end{columns}
	\vspace*{1em}
	\uncover<2->{
		\hspace*{-2em}
		\begin{tabular}{| c | c | c |}
			\hline
			Type of $S$ & Pros & Cons\\
			\hline
			Less, smaller sets & \specialcell{Less computational cost\\and memory footprint} & Less precision\\
			\hline
			More, bigger sets & More precision & \specialcell{More computational cost\\and memory footprint}\\
			\hline
		\end{tabular}\\
	}
	\vspace*{1em}
	\uncover<3->{
	{\large \color{blue} Examples:}\\
	\begin{itemize}
		\item $S_0 = $ all $2^n$ combinations of up to $n$ qubits
		\item $S_1 = $ all $\begin{pmatrix}
			n \\ k
		\end{pmatrix}$ combinations of exactly $k$ qubits
		\item $S_2 = $ sets that contain qubits used by at least two 3-qubit gates
		\item ...
	\end{itemize}
	}
\end{frame}

\begin{frame}{Choice of Tuple S (ctd.)}
	Another important factor in the choice for the abstract state is which qubits are put together in projections.\\
	A good choice would make the abstract interpretation \emph{flow-sensitive}.\\\;\\
	\pause
	{\large \color{blue} Example:}
	\begin{itemize}
		\item $U_1$ acts on qubits $\{1,2,3\}$ with $\{1,2\}$ as input and $\{3\}$ as output.
		\item $U_2$ acts on qubits $\{3,4,5\}$ with $\{3,4\}$ as input and $\{5\}$ as output.
	\end{itemize}
	\pause
	A good choice which would improve the precision of the abstract interpretation is to have the set $\{1,2,3,4,5\}$ in the abstract state.
\end{frame}

\begin{frame}{Assertion Checking Example}
	\uncover<+->{
		We want to check (partial) correctness of Grover's algorithm.\\\;\\
	}
	\uncover<2->{
		{\large \color{blue} Grover's algorithm steps:}
	}
	\begin{enumerate}[<+->]
		\item Prepare the initial state $\ket{\phi}^{(0)}$
		\item Repeat approx. $\frac{\pi}{4}\sqrt{N}$ times $\ket{\phi}^{(t+1)} = G\ket{\phi}^{(t)}$.
		\item Measure in the computational basis.
	\end{enumerate}\;\\

	\uncover<+->{
		With this abstract interpretation framework, we can prove the loop invariant:
		\begin{equation*}
			\ket{\phi}^{(t)} \in A = span\{\ket{\beta}, \ket{\phi}\}
		\end{equation*}
	}
\end{frame}


\begin{frame}{Assertion Checking Example (ctd.)}
	\begin{equation*}
		P(v) \triangleq v \in A = span\{\ket{\beta}, \ket{\phi}\}
	\end{equation*}
	\uncover<2->{
	We need to check:
	}
	\begin{enumerate}
		\uncover<2->{
			\item $P\left(\ket{\phi}^{(0)}\right) \uncover<3->{\triangleq \ket{\phi} \in span\{\ket{\beta}, \ket{\phi}\}}$
		}
		\uncover<4->{
			\item $P\left(\ket{\phi}^{(t)}\right) \implies P\left(\ket{\phi}^{(t+1)}\right) \uncover<5->{\,\,\triangleq\,\, \ket{\phi}^{(t)} \!\in\! A \,\,\implies\,\, G\ket{\phi}^{(t)} \!\in\! A}$
		}
	\end{enumerate}
	\uncover<6->{
		\begin{align*}
			G\ket{\phi}^{(t)} \!\in\! A = G\ket{\beta} \!\in\! A \land G\ket{\phi} \!\in\! A&& \quad\text{\footnotesize By linearity of G and } \ket{\phi}^{(t)} \!\in\! A
		\end{align*}
	}
\end{frame}

\section{Conclusions}
\begin{frame}{Future developments}
	\begin{itemize}
		\item Programs with measurements
		\item Conditionals
		\item Loops
		\item Mix of classical and quantum computation
		\item Choosing the optimal abstract space
		\item Other kinds of assertions
	\end{itemize}
\end{frame}

%MARK: Extra
\section*{Extra material}
\begin{frame}{Computing the Projection of a Support}[noframenumbering]
	To compute a projection corresponding to $supp(A)$, we:
	\begin{enumerate}
		\item take the rows $\{r_1, ..., r_n\}$ of A;
		\item extract an orthonormal set of vectors $\{b_1, ..., b_n\}$ that span the same subspace as the rows;
		\item create the matrix $B = \begin{pmatrix}
			b_1 \\ ... \\ b_n
		\end{pmatrix}$;
		\item return $BB^\dagger$.
	\end{enumerate}
\end{frame}

\begin{frame}{Computing the Projection of an Intersection}[noframenumbering]

    % //    (I \otimes ... \otimes I) -
    % //    supp( k(I \otimes I \times I) - (l.0 + ... + l.(k-1)) )
	\begin{gather*}
		\{P_1, ..., P_k\},\quad \forall i\in [k].\; P_i = C^n \times C^n\\
		\bigcap_{i\in [k]} P_i \triangleq I_n - supp(kI_n - \sum_{i\in [k]} P_i)
	\end{gather*}
\end{frame}

\begin{frame}{Computing the Assertion Projection}[noframenumbering]
	Given two vectors
	\begin{gather*}
		v_1 = \ket{a_1}...\ket{a_n}\\
		v_2 = \ket{b_1}...\ket{b_n}
	\end{gather*}
	\begin{enumerate}
		\item Create a matrix $P = \begin{pmatrix}
			v_1^T \\ v_2^T \\ \mathbf{0} \\ ... \\ \mathbf{0}
		\end{pmatrix}$
		\item Return $supp(P)$
	\end{enumerate}
\end{frame}

\begin{frame}{Partial trace}
	\begin{align*}
		Tr_B[\rho] &= \sum_{v=0}^{2^m} (I_A \tensor \bra{v})\rho(I_A \tensor \ket{v}) && Tr_A[\rho] = \sum_{v=0}^{2^n} (\bra{v} \tensor I_B)\rho(\ket{v} \tensor I_B)
	\end{align*}
	\begin{center}
		Where $v$ labels vectors of an orthonormal basis of the subspace we are tracing out.		
	\end{center}
\end{frame}

\begin{frame}{Weak Galois Connection}
	\begin{gather*}
		S \trianglelefteq T\\
		\forall \overline{\sigma}\in \mathit{AbsDom}(S).\; \forall \overline{\tau}\in \mathit{AbsDom}(T).\\
		\hspace*{-3em}
		\left(\begin{array}{@{}cc@{}@{}}
			\overline{\tau}\sqsubseteq \gamma_{S \rightarrow T}(\overline{\sigma}) \implies \alpha_{T\rightarrow S}(\overline{\tau})\sqsubseteq \overline{\sigma} \\
			\land\\
			\left(\exists \overline{\rho}\in\mathit{AbsDom}([n]^m).\;\overline{\tau} = \alpha_{[n]^m\rightarrow T}(\overline{\rho})\right) \;\implies\; \overline{\tau}\sqsubseteq \gamma_{S\rightarrow T}(\overline{\sigma}) \iff \alpha_{T\rightarrow S}(\overline{\tau}) \sqsubseteq \overline{\sigma} 
		  \end{array}\right)
	\end{gather*}
\end{frame}

\dospots
\addtocounter{framenumber}{-5}


% Introduction
% Reasons
% Ingredients
% Density matrices
% Abstract state space
% Galois connections
% Abstract operations
% Assertions
% Conclusions

%TODO: Check computations remembering that qubits are indexed rtl
%TODO: Underline that density matrices are projections
%TODO: Complete explanation of partial traces
%TODO: Maybe remove alternative formula for partial traces
%TODO: Go more into detail with the Linear Subspace frame
%TODO: Add theorems 6.1 and 6.2

\end{document}




% n_{qubits} = 5\\
% \ket{00000}\bra{00000}\\
% % \begin{pmatrix}
% % 	1 & 0 & 0 & 0 & 0 & 0 & 0 & 0 & 0 & 0 & 0 & 0 & 0 & 0 & 0 & 0 \\
% % 	0 & 0 & 0 & 0 & 0 & 0 & 0 & 0 & 0 & 0 & 0 & 0 & 0 & 0 & 0 & 0 \\
% % 	0 & 0 & 0 & 0 & 0 & 0 & 0 & 0 & 0 & 0 & 0 & 0 & 0 & 0 & 0 & 0 \\
% % 	0 & 0 & 0 & 0 & 0 & 0 & 0 & 0 & 0 & 0 & 0 & 0 & 0 & 0 & 0 & 0 \\
% % 	0 & 0 & 0 & 0 & 0 & 0 & 0 & 0 & 0 & 0 & 0 & 0 & 0 & 0 & 0 & 0 \\
% % 	0 & 0 & 0 & 0 & 0 & 0 & 0 & 0 & 0 & 0 & 0 & 0 & 0 & 0 & 0 & 0 \\
% % 	0 & 0 & 0 & 0 & 0 & 0 & 0 & 0 & 0 & 0 & 0 & 0 & 0 & 0 & 0 & 0 \\
% % 	0 & 0 & 0 & 0 & 0 & 0 & 0 & 0 & 0 & 0 & 0 & 0 & 0 & 0 & 0 & 0 \\
% % 	0 & 0 & 0 & 0 & 0 & 0 & 0 & 0 & 0 & 0 & 0 & 0 & 0 & 0 & 0 & 0 \\
% % 	0 & 0 & 0 & 0 & 0 & 0 & 0 & 0 & 0 & 0 & 0 & 0 & 0 & 0 & 0 & 0 \\
% % 	0 & 0 & 0 & 0 & 0 & 0 & 0 & 0 & 0 & 0 & 0 & 0 & 0 & 0 & 0 & 0 \\
% % 	0 & 0 & 0 & 0 & 0 & 0 & 0 & 0 & 0 & 0 & 0 & 0 & 0 & 0 & 0 & 0 \\
% % 	0 & 0 & 0 & 0 & 0 & 0 & 0 & 0 & 0 & 0 & 0 & 0 & 0 & 0 & 0 & 0 \\
% % 	0 & 0 & 0 & 0 & 0 & 0 & 0 & 0 & 0 & 0 & 0 & 0 & 0 & 0 & 0 & 0 \\
% % 	0 & 0 & 0 & 0 & 0 & 0 & 0 & 0 & 0 & 0 & 0 & 0 & 0 & 0 & 0 & 0 \\
% % 	0 & 0 & 0 & 0 & 0 & 0 & 0 & 0 & 0 & 0 & 0 & 0 & 0 & 0 & 0 & 0
% % \end{pmatrix}